\documentclass{article}
\usepackage{polski}
\usepackage[utf8]{inputenc}
\usepackage{amsmath}
\usepackage{amsthm}
\title{Wstęp do Fizyki Układów Złożonych}
\begin{document}
	\maketitle
	\newpage
	\pagenumbering{arabic}
	\section{Wykład 1}
		\subsection{Zasady zaliczenia}
		2 kolokwia po 10 pkt, trzeba przepołowić.
		\subsection{Aktualna tematyka Fizyki Układów Złożonych}
		\begin{itemize}
			\item zjawiska dynamiczne w sieciach złożonych
			\item zastosowania fizyki statystycznej do badania układów złożonych
			\item automaty komórkowe
			\item sztuczne sieci neuronowe
		\end{itemize}
		\subsection{Układy złożone - definicja}
		Są to układy złożone z dużej liczby elementów, które oddziaływują ze sobą. Są to układy zawsze o dynamice nieliniowej, pomimo tego obowiązują w niej uniwersalne prawa (np. prawa skalowania). Jeżeli chcemy to opisać, używamy obliczeń numerycznych i metod symulacji komputerowych - no bo to są układy nieliniowe! Przykłady: układy biologiczne, społeczeństwa mrówek, granulaty, sieci hydrologiczne, systemy ekonomiczne, internet, automaty komórkowe, układy charakteryzujące się pattern formation. 
		
		\subsection{Przykłady Układów Złożonych:}
		\begin{itemize}
			\item sieć neuronowa człowieka (mózg człowieka, $10^11$ neuronów, wiele połączeń pomiędzy nimi)
			\item społeczność ludzka (jednostki podzielone na grupy, połączenia interpersonalne, różne stopnie hierarchii), kształtowanie opinii, rozwój epidemii,  
			\item kwestie ekonomiczne i finansowe
			\item ziarna (piasek) \item lawiny ziemne albo śnieżne \item samoorganizująca się krytyczność
			\item biologia \item sztuczne życie
		\end{itemize}
	
		\subsection{Wybrane problemy Fizyki Układów Złożonych}
		\begin{itemize}
			\item Układy złożone w przyrodzie i metody ich badania
			\item Automaty komórkowe
			\item Sztuczne życie
			\item Samoorganizująca się krytyczność
			\item Inteligencja rozproszona
			\item Sztuczne układy społeczne
			\item Modelowanie wybranych zjawisk zachodzących w rzeczywistych ukłąach społecznych
			\item Zjawisko formatowania wzorców
			\item Sztuczne sieci neuronowe
			\item Szkła spinowe
			\item Podsumowanie wszystkich zjawisk i perspektywy rozwoju
		\end{itemize}
	
		\subsection{Przykłady użycia i zastosowania Fizyki Układów Złożonych}	
		\begin{itemize}
			\item Przykład neuronu.
			\item Przykład sieci internetowej.
			\item Przykład interakcji białek w drożdżach.
			\item Epidemie - ospa, hiszpanka, AIDS, SARS, ataki bronią biologiczną - teraz tworzy się modele opisujące te epidemie.
			\item Szczepionki mają duży wpływ na ograniczenie stopnia epidemii, jest jakiś śmieszny bezpieczny próg szczepień, dla niego następuje samotłumienie dla epidemii.
			\item Automaty komórkowe
			\item Pattern formation - powtarzanie się struktur, przykładowo liście lub naczynia krwionośne w płucach
			\item Satelity obserwują oceany, gdzie występuje zaburzenie pattern formation (powtarzalne fale), wtedy zostaje wychwycony np. statek albo jakiś kosmita.
			\item Przykład: badania płuc przesiewowe
			\item Przykład: zabarwienie muszli ślimaka, superpozycja dwóch pigmentów i silnych przesunięć fazowych pomiędzy nimi
		\end{itemize}
	\section{Wykład 2}
		Damian mi wyśle notatki.
	\section{Wykład 3}
		\paragraph{Właściwości lokalne CA} 
		\subparagraph{Najprostsza charakterystyka statystyczna CA} oznaczenia: binarna reprezentacja s liczby dziesiętnej \\
		Funkcja $\#_d(s) = \#_\alpha(n)$ - jest to liczba wystąpień cyfry $\alpha$ w binarnej reprezentacji s liczby n\\
		Przykład: $\#_1(182)=\#_1(10110110)=5$ \\
		$\#_0(182) = 3$\\
		średnia liczba 1 w automacie o dwóch wartościach {0,1} = $\rho_\tau$ po czasie ewolucji $t=\tau$ ($\rho_0$ - konfiguracja początkowa)\\
		ogólnie dla n z daszkiem jest $\#_1(n) + \#_0(n) = [log_2 n]$ - całkowita liczba cyfr w reprezentacji liczby binarnej\\
		... niedokończone
		 \paragraph{Problem} W chwili $t_0$ jest $\rho_t=\rho_0 \leftarrow$ gęstość jedynek \\
		 Jaka będzie dla danej reguły gęstość po czasie $t=\tau:\rho(t=\tau)$?\\
		 \paragraph{Odpowiedź} Dla dowolnej reguły jest to problem trudny, dla niektórych reguł jest trywialny, np. reguła 0 $\rightarrow \rho_2 = 0$ , regułą 254 $\rho_\infty=0$ itd. , to nie jest tak istotne
		 \paragraph{Prawidłowe wyniki dla reguły 90}
		 Zakładamy konfiguracje początkową : 0000000000100000000 \\
		 wprowadzamy oznaczenia ogólne: $N_\tau^{(1)} = 2^{\#_1(\tau)}$ - liczba jedynek w danym automacie po $t=\tau$\\
		 Można udowodnić że:\\
		 $\rho_\tau = \frac{1}{2}[1]$ ...\
		 \paragraph{Gęstość trójkątów}
		 w wielu regułach czesto pojawiają się trójkąty \\
		 $T_{(i)}(n)$ - Średnia gęstość trójkątów na podstawie n, wypełnienie cyfrą i\\
		 np. $T_(1) (5) = 1$\\
		 $T_(0) (5) = 2$ \\
		 $T_(1) (7) = 1$ \\
		 Jest wiele obliczeń analitycznych dla początkowych reguł np. dla reguł 150 otrzymano $T_(1) (n) = n^{-log_2(2\phi)}=n^1,69$ gdzie $\phi = \frac{1}{2}(1+\sqrt{5})$
		 \paragraph{Gęstość sekwencji}
		 n jednakowych cyfr (i) ograniczonych cyframi innymi (np. dla i=1 są to zera) - $Q_{(i)} (n)$ \\
		 $Q_{(i)}(n) = \Sigma[\frac{2T_(i)(j)}{j}]$ 
		 \subsubsection{Automaty komórkowe probabilistyczne}
			 Reguły lokalne dynamiki uwzględniają szum. Formalnie odpowiada to wprowadzeniu temperatury. \\ 
		\subsection{Sztuczna inteligencja}
			Bakteria - twór którego funkcjonowanie opiera się na złożonych reakcjach biochemicznych \\
			Komputer - analogia do istot żywych, ale to nie jest ciągle życie \\
			\textbf{Istota żywa} wg Arystotelesa to układ posiadający zdolność do reprodukcji, układ pobiera informacje z otoczenia i komunikuje się z nim\\
			Układ żyjący charakteryzuje się wzrostem organizmu na wskutek metabolizmu, reprodukcja, modyfikacja na skutek działania z otoczeniem \\
			Schroedinger - życie nie jest czymś co można wyjaśniać tylko na podstawie badań fizycznych, potrzebne jest coś ekstra \\
			Każdy byt jest niezależną monadą która ma własny rytm i mechanizm życia - Leibniz
			\paragraph{Początki} W Los Alamos była pierwsza konferencja na temat SI w 1987 - workshop of artificial life. 
			\textbf{Chrostopher Langton} - założyciel. 
			Wnioski - nie jest możliwe stworzenie prawdziwego życia, możemy tworzyć obiektu posiadające cechy obiektów żywych. Możemy otrzymać zjawiska dla obiektów żyjących: ewolucja, rozmnażanie, rozpad, współzawodnictwo. Istnieje możliwość stworzenia rzeczywistych układów żyjących, udało już się stworzyć sztuczną bakterię, sklejenie jej z elementów innych bakterii.\\
			\paragraph{Artificial Life:}
			\begin{itemize}
				\item badania biologiczne
				\begin{itemize}
					\item badania biologiczne
					\item pochodzenie życia
					\item badanie ewolucji
					\item synteza DNA/RNA
				\end{itemize}
				\item prawa inteligentnego zachowania
				\begin{itemize}
					\item samoorganizacja
					\item zachowania grupowe
					\item roboty autonomiczne
					\item inteligencja rozproszona
				\end{itemize}
			
				\item zastosowania praktyczne
				\begin{itemize}
					\item wykorzystanie komputerów do układów sztucznego życia
					\item wykorzystanie do gier
					\item optymalizacja
				\end{itemize}
			\end{itemize} 
			
			\paragraph{Pętla Langtona} Źródło, jakieś czasopismo fizyczne w 1984 roku, został przedstawiony automat który potrafi się reprodukować, może być odzwierciedleniem żyjących organizmów. Układ składający się ze zbioru reguł które się poruszają i się reprodukują. Pętla wypuszcza wypustkę i z niej powstaje nowa pętla. Po jakimś czasie powstaje cała kolonia, w środku są pętle bez wypustek - obumarłe.
			\paragraph{Sąsiedztwo} Mamy dwa rodzaje sąsiedztwa: \textbf{Moore'a} (8 komórek na około) i \textbf{von Newmanna} (4 komórki bez kątów)
		\section{Wykład 4}
			\paragraph{Pętla Langtona - ciąg dalszy} Gen śmierci anihiluje komórki które nie mogą się rozmnażać. Jest to układ który się rozrasta i potem się rozmnaża. Potrzebuje on przestrzeni życiowej. Jeżeli w swoim rozwoju napotka na przeszkodę, nie może się dalej rozwijać. Osobniki mniejsze są bardziej przystosowane do przeżycia. 
			\paragraph{Game of Life (GoL)} Conway (1970) jest jej pomysłodawcą. Automat komórkowy który jst intensywnie badany, jest to automat gdyż posiada on ścisłym regułom. Sąsiedztwo wyznacza ewolucje stanu komórki. Przykład automatu totalistycznego - stan komórki zależy od sumy stanu komórek sąsiednich. Można zauważyć wiele cech sztucznego życia. Czy to są układy deterministyczne? Są to na pewno procesy nieodwracalne. 
			\paragraph{Reguły GoL}
			\begin{itemize}
				\item stany 0 i 1 (dead/alive)
				\item jeżeli żyjąca komórka ma mniej niż dwóch sąsiadów żyjących umiera z \textbf{samotności}
				\item jeżeli komórka ma więcej niż 3 sąsiadów żyjących umiera z \textbf{tłoku}
				\item jeżeli ma dokładnie 3 sąsiadów żywych to wtedy \textbf{ożywa}
				\item w pozostałych przypadkach zostaje bez zmian.
			\end{itemize}
			\paragraph{Typy różnych bloków w GoL}
			\begin{itemize}
				\item \textbf{Typ I} - wciąż żyjące, nie zmieniają się, są stałe (block, tub, snake, integral)
				\item \textbf{Typ II} - oscylatory 
				\item \textbf{Typ III} - statki kosmiczne, wzór który po n krokach przesuwa się w jakąś stronę
				\item \textbf{Typ IV} - wzory które ciągle zwiększają swój rozmiar
				\begin{itemize}
					\item \textbf{Typ a} - guns - co jakiś czas produkują nowy element
				\end{itemize} 
			\end{itemize}
		 
		 Oddziaływania lokalne z najbliższymi sąsiadami odgrywają fundamentalną rolę w automatach komórkowych.
		\paragraph{Jakie cechy życia są widoczne w GoL?} 
		 \begin{itemize}
		 	\item niszczący wpływ środowiska
		 	\item ewolucja
		 	\item śmierć
		 	\item ruch
		 	\item rozmnażanie
		 \end{itemize}
	\section{Wykład 5}
		\paragraph{}Karl Sims - Virtual Creature. Tworzenie przy pomocy symulacji komputerowych trójwymiarowych sztuczne organizmy współzawodniczące pomiędzy sobą w różnych dziedzinach - skakanie, bieganie, itp. 
		Można generować takie stworzenia dowolnie, czasem przybierają podobne formy do zwierząt, składają się one z prostopadłościanów złożonych ze sobą, są pewne ograniczenia tak jak w stawach, wykazują cechy prawdziwych zwierząt, są też nowe formy. 
		
		\subparagraph{XIX wiek} Kreacjonizm - życie powstało z woli Boga, autogonia - powstanie spontaniczne organizmów z materii. Lata dwudzieste - Oparin - w pierwotnych warunkach na ziemi istniało morze, wielkie zbiorniki wodne, w tych zbiornikach były związki chemiczne, $CO_2$, amoniak, formaldehyd, występowały silne i częste wyładowania elektryczne, i wtedy powstały oddzielne pęcherzyki, które miały złożony skład chemiczny, i one były zdolne do wzrostu i podziału - pierwociny komórki. 
		\subparagraph{Teraz} Badanie życia, w najprostszych organizmach eliminuje się kolejne geny i sprawdza się czy komórka dalej się rozmnaża. Ten ziomeczek uważa żę jest możliwe że stworzymy nowe organizmy. R. Breslow, Columbia University, rozważa możliwość że życie pochodzi z kosmosu. Te śmieszne meteoryty przyniosły związki organiczne - aminokwasy które stanowiły początek organizmów żywych na ziemi. Ale: wszystkie stworzenia na ziemi składają się z aminokwasów lewoskrętnych a jak się dołoży prawoskrętne to wtedy następuje rozkład. To jest pewien dowód na to że to lewoskrętne aminokwasy dotarły na ziemię. Tak asteroida w czasie podróży w kosmosie trafiła w sąsiedztwo gwiazd neutronowych, duże pole magnetyczne, aminokwasy prawoskrętne zostały zniszczone, zostały tylko lewoskrętne. 
		\paragraph{Zupa ze złożonych związków chemicznych} Mamy roztwór w którym powstają twory skłądające się z głowy i z ogona. Na początku mamy pojedyncze molekuły, żeby powstało cos złożonego, np. jakieś micele.
		\subparagraph{}L. Edwards, Y. Peng, AL, \underline{6}, 35-72(1998)	Symulacja wielu cząsteczek w roztworze wodnym.
		\paragraph{} Siły: 
		\begin{itemize}
				\item $f_1$ - przyciagająca, head-head attraction 
				\item $f_5$ - tail-tail attraction 
				\item $f_3, f_4$ - head-tail, tail-head repulsion 
				\item $f_2$ - repulsion
				\item $f_6, f_7$ - sztywność molekuł 
		\end{itemize}
		Moment sił:
		$$T_i(t)=\Sigma_{i\neq j}[\Sigma \tau_K(t)]+rand_2(t)-c_2u_2(t)$$
 		$$r=\frac{1}{6}l$$
 		$$m=1$$
		 Środowisko składało się z 900 komórek, w każdej z nich mogła być cząsteczka, r=5, l=30, w tym środowisku umieszczono przypadkowo 200 cząstek, uruchomiono symulację.
		 \paragraph{Próby stworzenia prawdziwego życia} Jest kilkaset zespołów które starają się stworzyć sztuczne życie. W NY stworzono sztuczny wirus choroby polio, ten sztuczny wirus sam się replikował. Na uniwersytecie Rockeffelera stworzono sztuczną bakterię z różnych rzeczy, jej świecenie świadczy o tym że żyje itp. Zasób informacji potrzebny do powstania bakterii to książka średniej grubości (w przypadku wirusa to jedna strona). Trzeba pamiętać o tym że można stworzyć bakterię która może zabić dużo ludzi.
		 \paragraph{Co da nam sztuczne życie?} Usuwanie złogów w naczyniach krwionośnych, poruszanie się w układzie pokarmowym człowieka i np. redukcja kwasu solnego żołądku. 
	\section{Wykład 6}
	\section{Wykład 7}
		\paragraph{Podstawowe modele epidemiologiczne}
		\begin{itemize}
			\item SIR (susceptible/ill/removed) 
			$$\dot{S}=BIS$$
			$$\dot{I}=\beta IS - \gamma I$$
			$$\dot{R} = \gamma I$$
			$$S+I+R = const$$
			\item SEIR (+ exposed)
			\item SIS (susceptible/ill/susceptible) - jest gorszy, można kilkakrotnie być chorym
			\item SEIRS (susceptible/exposed/ill/removed/sussceptible)
		\end{itemize}
		To są modele w których nie zakłada się w ogóle kontaktów strukturalnych.
		Zastosowanie liczby parametrów w opracowanym modelu umożliwia dostosowanie go do:
		\begin{itemize}
			\item populacji różnych wielkości
			\item kontaktów społeczne
			\item infekcji o różnych właściwościach			
		\end{itemize}
		
		Zasięg epidemii - im więcej jest szczepień, tym większe prawdopodobieństwo że epidemia się nie rozszerzy, ie przekroczy pewnego punktu. \\
		Szczepienia celowane - szczepienie jednostek najbardziej podatnych na epidemie,  mają najgęstszą sieć kontaktów interpersonalnych, np. służby celne czy służby medyczne.
		\subparagraph{Perkolacja} Zjawiska występujące w różnych materiąłach, mamy stop który jest reprezentowany przez sieć regularną, jesłi w tym stopie będą skłaanidkami szare i czarne kropki, i czarne kropki się ze sobą łączą, i to może prowadzić do przepływu prądu. Sperkolowany - przewodący ?. Jest to taka sytuacja że możemy znaleźć drogę prowadzącą od jednej krawędzi do drugiej.
		\subparagraph{}To samo zachodzi w przypadku szczepień - powyżej pewnej ilości szczepień nie zachodzi już epidemia - bezpośredni związek z perkolacją.
		\paragraph{KOLOKWIUM: 29.11}
		\begin{itemize}
			\item Podaj definicję automatów komórkowych i podział na klasy wg Wolphrama
			\item Co to jest Game Of Life? Jakei cechy życia są obserwowane w GOL? NArysować 3 następne kroki czasowe.
			\item Model epidemiologiczny SIR (Stany jednostek, równania różniczkowe, dwie krzywe: zasięg i krzywa epidemiologiczna)
		\end{itemize}
			
			
		
		
		 
		 
		 
		 
		 
		 
		 
		
		
		

		
\end{document}