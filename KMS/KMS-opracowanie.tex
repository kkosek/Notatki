\documentclass{article}
\usepackage{polski}
\usepackage[utf8]{inputenc}
\usepackage{amsmath}
\usepackage{amsthm}
\title{KMS - opracowanie}
\author{Adam Szczepański, Katarzyna Kosek}

\begin{document}
		\maketitle
		\newpage
		\pagenumbering{arabic}
	\section{Wykład 1}
		\subsection{MD - cechy podstawowe metody, zastosowania}
		Metody dynamiki molekularnej służące do obliczania trajektorii przestrzeni fazowej zbioru molekuł z których każda polega klasycznym równaniom.
        \paragraph{Cechy:} 
        \begin{itemize}
            \item klasyczna metoda: 
            $$H = \Sigma_{i=1}^N\frac{p_i^2}{2m_i} + \Sigma_{i,j}u(V_{ij})$$
            \item postać oddziaływania - oddziaływanie klasyczne gdzie $r_c$ to odległość odcięcia - konwencja najbliższego obrotu
            \item periodyczne warunki brzegowe
            \item efekty wymiaru
        \end{itemize}
        \paragraph{Zastosowania}
        \begin{itemize}
            \item symulacja procesu topnienia gazu szlachetnego
            \item modelowanie układów biologicznych
            \item wyznaczanie wielu średnich i chwilowych wiekości fizycznych  badanych układach
        \end{itemize}
        \subsection{Algorytm Verleta}
        \begin{itemize}
	        \item jest odwracalny w czasie
	        \item zachowuje strukturę symplektyczną dynamiki Hamiltonowskiej
	        \item całkowita energia układu fluuktuje wokół jakiejś wartości
	        \item szybki, ale niezbyt dokładny
	    \end{itemize}
	    Algorytm podstawowy:
	    $$r(t + \Delta t) = 2r(t) - r(t - \Delta t) + \frac{f(t)}{m}\Delta t^2$$
	    $$v(t) = \frac{2r(t) - r(t - \Delta t)}{2\Delta t}$$
        Przykład: układ monoatomowy z oddziaływaniem typu równanie L-J:
        $$U(r_{ij}) = 4\epsilon((\frac{\sigma}{r_{ij}}^{12} - \frac{\sigma}{r_{ij}}^{6	}))$$
		Część przyciągająca, część odpychająca.
		\subsection{Algorytmy MD dla zespołów NVE, NVT, NPH}
		\begin{itemize}
			\item NVE (mikrokanoniczny)	- stałe N, V, E
				\paragraph{Postać algorytmu}
				\begin{itemize}
					\item określenie położeń $r_i^0, r_i^1$
					\item obliczanie sił dla n-tego kroku $F_i^n$
					\item obliczenie położeń dla n+1 kroku ze wzoru
					\item obliczenie prędkości dla n+1 kroku 
					\item obliczenie wielkości termodynamicznych
				\end{itemize}
				$$E = const$$
				$$T = const$$
				$$E_C = E_k + U$$
				średnia po trajektorii w takim układzie:
				$$\bar{A} = <A>_{NVE} = \lim\limits_{t \rightarrow \infty} 
				(t' - t_0)^{-1} \int_{t_0}^{t'}dtA(\vec{r}^N(t),\vec{p}^N(t),V)$$
				$$\bar{E_k} = \lim\limits_{t \rightarrow \infty}(t' - t_0)^{-1}
				\int_{t_0}^{t'}E_k(v(t))dt$$
				$$\bar{U} = \lim\limits_{t \rightarrow \infty}(t' - t_0)^{-1}
				\int_{t_0}^{t'}U(v(t))dt$$
				Korzystając z zasady ekwipartycji energii:
				$$\frac{1}{2}mV^2 \sim \frac{1}{2} k_BT$$
				$$E_C = E_K + U$$
				Organizacja symulacji: 
				\begin{itemize}
					\item inicjowanie - warunki początkowe
					\item doprowadzenie do równowagi
					\item realizacja wg zadanego algorytmu (liczenie hamiltonianu taki naprawdę podstawowy)
				\end{itemize}
			\item NVT (kanoniczny)
					\begin{itemize}
						\item T = const
						\item należy odpowiednio zmodyfikować równania ruchu
						$$<H>_{NVT} = \lim\limits_{t' \rightarrow \infty } 
						\int_{t_0}^{t'}H(\vec{r}^N(t), \vec{p}^N(t), V)dt$$
						\item należy dodać termostat w taki sposób aby bieżące średnie wartości różnych obserwabli były liczone w odpowiednim zespole statystycznym , jest wiele możliwych realizacji uwzględnienia termostatu
						\item w niektórych krokach zmienia się prędkość (jeżeli występuje to zbyt rzadko, to nie osiągnie się temperatury końcowej, jeżeli za często następuje błądzenie przypadkowe)
						\item dodanie siły wymuszającej
					\end{itemize}
				\paragraph{Algorytm}
				\begin{itemize}
					\item inicjalizacja położeń i prędkości
					\item nowe położenia
					$$r^{n+1} = r^n + nv_i^n  + n^2 \frac{F_i^n}{2m}$$
					\item obliczenie $\Sigma_i^{n+1}mv^2$
					\item przeskalowanie prędkości
				\end{itemize}
			\item NPH (izobaryczno - izoentalpowy)
			\begin{itemize}
				\item układ w którym temperatura i ciśnienie są ustalone
				\item całkowita energia wewnętrzna nie jest zachowana
				\item zachowana jest entalpia
				$$<H>_{NPH} = \lim\limits_{t' \rightarrow \infty}(t' - t_0)^{-1}\int_{t_0}^{t'}dtH(p^N, v^N, V)$$
			\end{itemize} 
			\paragraph{Algorytm}
			\begin{itemize}
				\item inicjalizacja
				\item określ objętość początkową 
				\item określ początkową prędkość zmian odległości
				\item oblicz $\rho ^ {n+1}$ oraz $V^{n+1}$
				\item oblicz prędkości cząstkowe dla cząstek i objętości 
				\item oblicz siły i potencjalną część wi..
				\item oblicz ciśnienie $p^{n+1}$ na podstawie prędkości cząstkowych
				\item oblicz prędkość zmian objętości
				\item oblicz prędkość cząstek na podstawie prędkości cząstkowych
			\end{itemize}
		\end{itemize}
		\section{Wykład 2}
		\subsection{energia całkowita układu molekularnego w ujęciu klasycznym}
		NIc o tym nie znalazłem :(
		
		\subsection{tw. o wiriale (Twierdzenie Clausiusa)}
		-opisuje ono zależność pomiędzy średnią energią potencjalną a średnią energia kinetyczną cząstki lub układu.\newline	
		- zgodnie z nim dla pojedynczej cząstki poruszającej się ruchem ograniczonym w polu $V=ar^{n}$,średnia energia spełnia zależność:\newline	
		\begin{center}
			2$<E_{k}>=n<E_{p}>$\newline
		\end{center}
		-zastosowanie w fizyce statystycznej\newline	
		-pozwala obliczyć średnią kinetyczną układu bez analizowania ruchu pojedyńczych
		cząstek\newline					
		Przykłady:\newline		
		a) oscylator harmoniczny $V=k{r}^{2}$\newline		
		\begin{center}
			$<E_{k}> = <E_{p}>$ (dwójki się skracają)\newline	
		\end{center}		
		b) planeta w polu grawitacyjnym $V=-\frac{k}{r}$\newline
		\begin{center}
			$2<E_{k}>=-<E_{p}>$\newline
		\end{center}	
		
		\subsection{dynamika brownowska	– podstawowy opis}
		Dynamika brownoska. Metoda BD polega na obliczaniu trajektorii w przestrzeni fazowej zbioru molekuł, z których każda spełnia równania Langevina w polu sił. Podobnie jak w przypadku metody MD, za pomocą metody BD możemy opidywać zarówno cząski punktowe, jak i zbiory cząstek złożonych z podjednostek. Trajektoria ruchu cząstki jest wyznaczana w oparciu o algorytm Ermaka i McCammona:\newline
		\begin{center}
			$m\frac{dv(t)}{dt}=R(t)-\beta v(t)$\newline
		\end{center}
		-siła losowa znika w uśrednieniu $<R(t)>=0$\newline
		-siły są nieskorelowane dla 2 różnych czasów $<R(t),R(t')>=q\delta (t-t')$\newline
		
		\section{Wykład 3}
		\subsection{metoda MC - podstawy}
		Metoda Monte Carl (MC)- polega na przedstawieniu rozwiązania postawionego problemu w postaci parametru pewnej hipotetycznej populacji i używaniu losowej sekwencji liczb do tworzenia próbki tej populacji, na podstawie której można dokonać statystycznego oszacowania wartości badanego parametru\newline
		-Duży zakres zastosowań
		
		\subsection{metoda MC dla zespołu mikrokanonicznego}
		Przy symulacji metodą MC dla zespołu mikrokanonicznego, w hamiltonianie jest opuszczany człon odpowiadający energii kinetycznej. A zatem, obliczając własności układu, nie możemy stosować równań ruchu. Przyjętym podejściem jest wyznaczenie własności układu przy użyciu sumy stanów Z. W przypadku rozpatrywanego przez nas układu zachowawczego z liczbą cząstek N i daną liczbą V, rozkład dla zespołu mikrokanonicznego jest opisywany przez funkcje $\delta$ ,a zatem syme stanów można opisać następująco :\newline
		\begin{center}
			$Z=	\int_{\Omega}^{} \delta (H(x)-E)dx$
		\end{center}
		gdzie E- ustalona energia układu, H- hamiltonian układu (ograniczony do E)
		Z każdą obserwablą A jest związana funkcja A(x), która zależy od stanu układu. Zakładamy że wartość obserwabli A jest równa średniej po zespole :
		\begin{center}
			$<A>_{NVE}=\frac{1}{Z}\int_{\Omega}^{} A(x)\delta(H(x)-E)dx$\newline
		\end{center}
		Robimy jakieś sztuczki i otrzymujemy(jeśli ktoś wie jakie niech napisze) :
		\begin{center}
			$Z=\sum_{x} \sum_{E_{D}} \delta(H(x)+E_{D}-E)dx$\newline
		\end{center}
		
		Algorytm NVE Monte Carlo :\newline
		\begin{enumerate}
			\item Utworzy stan dla którego H(x)=E
			\item Zadaj energie demona $E_{D}(np E_{D}=0)$
			\item Wybierz jakąś część układu
			\item Zmień tak stan lokalny układu, aby $x->x'$
			\item Oblicz wytworzoną w ten sposób zmianę entropii, tj.\newline
			\begin{center}	$\Delta H=H(x')-H(x)$ \end{center}
			\item Jeżeli nastąpiło obniżenie energii, to zaakceptuj zmianę, przyjmij że $E_{D}-< E{D}-\Delta H $ i traktuj x' jako nową konfigurację.
			\item Wróć do punktu 3.
		\end{enumerate}
		W końcowym rezultacie energie podlegają rozkładowi Boltzmanna co umożliwia policzenie energii:
		\begin{center}
			$P(E_{D})\infty exp(-\frac{E_{D}}{K_{B}T}))$ ( $\infty$ była obcięta)
		\end{center}
		\subsection{obliczanie całek metodą MC}
		Jednym z zastosowań metod Monte Carlo jest wyznaczanie przybliżonej wartości całki danej funckij f(x) w zadanym przedziale [A,B]. W tym celu:\newline
		\begin{enumerate}
			\item Sporządzamy wykres funkcji w przedziale [A,B] i na jego podstawie wyznaczamy prostokąt o podstawie równej przedziałowi całkowania [A,B] a wysokości dowolnej byle nie mniejszej od maksimum funkcji w tym przedziale,
			\item generujemy w tym prostokącie wiele punktów poprzez losowanie ich współrzędnych według rozkładów równomiernych o zakresach odpowiadających bokom prostokąta 
			\item wyznaczamy przybliżoną wartość całki według następującej proporcji:
			"stosunek pola pod krzywą (czyli całki) do pola prostokąta jest równy stosunkowi liczby punktów pod krzywą do liczby wszystkich wylosowanych punktów w tym prostokącie". 
		\end{enumerate}
		\section{Wykład 4}
		\subsection{Rozwiązywanie r. Schroedingera na siatce dyskretnej –	przykład rozwiązania zagadnienia rozpraszania}
		\begin{center}
			$ \Psi (r) \lambda e^{ikr}+f(\theta)\frac{e^{ikr}}{r}$ \\
			$\frac{d\sigma}{d\Omega}=|f(\theta)|^{2}$ \\
			
			całość jest na stronie zbereckiego materiały: rozpraszanie
		\end{center}
		\subsection{rozwiązywanie r. Schroedingera przez sprowadzenie do algebraicznego zagadnienia własnego}
		\begin{center}
			$ \hat{H} \Psi = E\Psi $  (*) \\
			$ \hat{H} = -\frac{\hbar}{2m}\nabla ^{2}+V(x)$\\
			$ \tilde{V}=\frac{V}{\frac{\hbar}{2m}}$ $\tilde{E}=\frac{E}{\frac{h^{2}}{2m}}$ -jednostki bezwymiarowe \\
			$\tilde{H}=-\frac{d^{2}}{dx^{2}}+\tilde{V}(x) $ oraz   $|\Psi> = \Sigma _{i=1}^{\infty} \frac{C_{i}}{\phi _{i}}$ (*2)\\	
		\end{center}
		po wstawieniu (2*) do (*) i pomnożeniu z lewej przez <p|(dalej nie widac). Dostajemy układ  N równań\\
		\begin{center}  
			$<\phi_{1}|H\phi_{1}> C_{1} + <\phi_{2}|H\phi_{2}> C_{2} +..... = E C_{1}$(chyba)\\
		\end{center}
		Otrzymujemy układ $|\Psi>= \Sigma_{i}^{N} C_{i}|\psi>$ N-równań  na współczynniki $C_{i}$ co sprowadza się do algebraicznego zagadnienia własnego \\
		\begin{center}
			$H\vec{C}=E\vec{C}$ gdzie H- macierz hamiltonianu i $\vec{C}=[C_{1},C_{2},....,C_{N}]$ \\
			$ \tilde{H}=\int \phi_{i} \tilde{H} \phi_{i}dx$\\
		\end{center}
		Z równania 	$H\vec{C}=E\vec{C}$(równanie macierzowe) otrzymujemy N wartości $E_{1},....,E_{N}$ natomiast energią stanu podstawowego jest minimum z tych wartości $E_{min}$.
		\subsection{rozwiązywanie r. Schroedingera na siatce dyskretnej – schemat Numerova i metoda strzałów}
		
	\section{Wykład 5}
		\subsection{Metoda wariacyjna dla oscylatora}
        Polega ona na tym, że należy zgadnąć funkcję falową która najlepiej przybliża funkcję falową stanu podstawowego. Potem za pomocą odpowiednich przekształceń dochodzimy do rozwiązania 
        Funkcjonał energii:
		    $$F[\Psi] = E - \frac{\int_{-\infty}^{+\infty}\Psi^*H\Psi dr^3}{\int_{-\infty}^{+\infty}\Psi^*\Psi dr^3}$$	
		    
		    $$E_{\text{stanu cząstki}} = \delta F[\Psi] = 0 $$ gdyż energia stanu podstawowego dla wariacji funkcjonału wynosi 0
		    $$P = <\Psi | H | \Psi >$$
		    $$Q = <\Psi, \Psi>$$
		    $$E = \frac{P}{Q}$$
		    $$<H> = \int_{-\infty}^{+\infty} \Psi^* H \Psi d^3 r$$
		    $$\delta E = 0 = \delta\int_{-\infty}^{+\infty} \Psi ^* H \varphi d^3r$$
			$$\hat{H} = \hat{T} + \hat{V} = -\frac{\hbar^2}{2m}\nabla + \frac{m\omega^2}{2}r^2$$
			Funkcja próbna:
			$$\varphi = A(1 + \alpha r)e^{-\alpha r}$$
			Z hamiltonianem (1) i funkcją próbną wyrażenie na E jest funkcją parametru $\alpha$. po obliczeniach mamy:
			$$<H> = H(\alpha) = \frac{3}{14} \frac{\hbar ^2 \alpha ^2}{m} + \frac{m \omega^2  }{\alpha^2}\frac{81}{21}$$
			Różniczkujemy równanie i rozwiązujemy równanie na $\alpha$:
			$$H_{min} = ? \implies \frac{dH(\alpha)}{d\alpha} = 0$$
			$$\alpha_{\min} = 3 \sqrt{\frac{3}{2}}\frac{m\omega}{\hbar}$$
			czyli
			$$H(\alpha_{\min})\approx1,57 \hbar \omega$$
			Czyli całkiem ok (powinno być 1,5). Z zasady wariacyjnej można uzyskać tylko energię stanu podstawowego i sam stan podstawowy. Stany wzbudzone można otrzymać innymi metodami.
		\subsection{Metoda wariacyjna dla cząsteczki $H^+$}
			$$H = -\frac{1}{2} \nabla^2 - \frac{1}{v_1} - \frac{1}{v_2}$$
			Wybierzmy bazę: jest to baza funkcji gaussowskich.
			$$\alpha \in <\alpha_1, \alpha_2>$$
			$$\varphi(\vec{r}) = \Sigma^N_{i=1, j=1} c_{ij}f_{ij}(\vec{r})$$
			$$f_{ij} = exp(-(\frac{\alpha}{i^3})r_1^2 - \frac{\alpha}{j^3}r_2^2)$$
			Należy zbudować macierz hamiltonianów i jeśli nie mamy bazy ortogonalnej to musimy rozwinąć zagadnienie w postaci uogólnionego zagadnienia rozproszonego.
			$$H_c = ES_c$$
			gdzie S to macierz overlapów, a $s_{ij}$ to funkcje bazowe bez hamiltonianów
			Nasza super macierz:
		$$	\begin{array}{cccc}
				<11|H|11> & <11|H|12> & <11|H|21> & <11|H|22> \\ 
				<12|H|11> & <12|H|12> & <12|H|21> &  <12|H|22>\\ 
				<21|H|11> & <21|H|12> & <21|H|21> &  <21|H|22>\\ 
				<22|H|11> & <22|H|12> & <22|H|21> &  <22|H|22>\\ 
			\end{array} $$
			
			Minimalizujemy $E_c$.
			Stosujemy regułę wariacyjną, funkcja falowa jest przewidywana w postaci:
			$$\varphi(\vec{r_1}S_1, \vec{r_2}S_2)$$
			Ej na serio nie wiem jak to dokończyć. To jest naprawdę okropne.
			\subsection{Przybliżenie B-O i przyblizenie pola średniego}
			W przybliżeniu Bohra-Oppenhaimera wykorzystuje się fakt, że jądra są o wiele cięższe od elektronów co czyni je o kilka rzędów wielkości wolniejszymi. Sprawia to, że elektrony błyskawicznie przystosowują się do ruchu jąder przez co można rozseparować jądrowe i elektronowe stopnie swobody. Z matematycznego punktu widzenia funkcję falową można rozłożyć na część elektronową i jądrową błyskawicznie przystosowują się do ruchu jąder przez co można rozseparować jądrowe i elektronowe stopnie swobody. Z matematycznego punktu widzenia funkcję falową można rozłożyć na część elektronową i jądrową
            Przybliżenie B-O stosuje się do uzyskania energii potencjalnej, dzięki niej można stosować fizykę klasyczną.
            Mamy cząsteczkę dwuatomową o N elektronach. 
            $$\hat{H} = -\frac{\hbar^2}{2M_a}\Delta a - \frac{\hbar^2}{2M_b}\Delta b - \frac{\hbar^2}{2m} \Sigma^n_{i=1} \Delta i	+ \hat{V}$$
            Dwa pierwsze człony - energia kinetyczna dowolnych jąder, ten z sumą to energia kinetyczna elektronów, V to energia oddziaływań między cząstkami.
            \subparagraph{}Wprowadzamy odległość między nimi:
            $$\hat{H} = \hat{H_0} + \hat{H}'$$
            $$\hat{H_0} = -\frac{\hbar^2}{2m} + \Sigma_{i=1}^n \Delta r_i + V$$
            $$\hat{H}' = -\frac{\hbar^2}{2\mu}\Delta R$$
            Gdzie $r_i$ to odległości elektronów od jąder.
            \subparagraph{}Elektronowe równanie Schrodingera niezależne od czasu:
            $$\hat{H}_0\Psi(\vec{r}_1, \vec{r}_2, ... , R) = E_k^0(R)\Psi_k(\vec{r}_1, \vec{r}_2, ... , R)$$
            Możemy wprowadzić funkcję falową całej cząstki:
            $$\Psi = \Psi_k(\vec{r}_1, \vec{r}_2, ... , R)\Phi_k(R)$$
            Równanie opisujące ruch jąder ma postać:
            $$\hat{H_{ef}}\Psi_k(R) = k\Psi_k(R)$$
            gdzie $H_{ef} = E_k^0(R) + H_{kk}'(r)$
			gdzie to Hkk to wpływ jąder na ruch elektronu, i to pomijamy i to ejst to przybliżenie.
					\section{Wykład 6}
					\subsection{metoda HF – podstawy opisu}
					Metoda Hartree-Focka - służy do rozwiązywania wielociałowego układu fermionów. Funkcja falowa takiego układu musi spełniać pewne warunki miedzy innymi antysymetrie.\newline
					Szkic wyprowadzenia\newline
					\begin{enumerate}
						\item Postulujemy pewną postać funkcji falowej $\phi$
						\item Wstawiamy zapostulowaną postać $\Phi$ do wyrażenia na całkowitą energie E
						\item Rozpisujemy wariancje $\delta$E =0 ze względu na zmianę orbitali $\Psi$ uwzględniając warunki Lagrange'a
						\item Otrzymujemy równania jednocząstkowe HF
					\end{enumerate}
					Funkcja falowa ma postać wyznacznika Slatera :
					
					$\phi =\frac{1}{\sqrt{n!}} \left[
					\begin{array}{ccccc}
					\phi_{1}(r_{1}) & \phi_{1}(r_{2}) &... &\phi_{1}(r_{n}) &\\
					\phi_{2}(r_{1}) & \phi_{2}(r_{2}) &... &\phi_{2}(r_{n}) &\\
					...             &     ...         &... &...           &\\
					\phi_{n}(r_{1}) & \phi_{n}(r_{2}) &... &\phi_{n}(r_{n}) &\\
					\end{array}
					\right]
					\qquad 
					$	
					gdzie $\phi_{i}(r_{i})$ spinorbitale \newline
					
					Zakładając ,że sponorbital jest absorbowany przez 2 elektrony :\newline
					\begin{center}
						$\Psi_{i}(r_{j})=\Psi_{i}(j) $ gdzie $\phi_{i}(j)$ $\Psi_{i}(j)$ (nie widzę dokładnie)
					\end{center}
					
					\subsection{metoda HF –	algorytm rozwiązania}
					Równanie HF rozwiązuje się algorytmem samozgodnym(chyba nie widzę dokładnie)  wprowadza się średni potencjał:
					\begin{center}
						$\tilde{V}(i)= \Sigma_{q=1}^{n/2} [2 \tilde{J}_{q}(i)-\tilde{K}_{q}(i)]$\\
						$\tilde{F}(i)=\tilde{h}(i)+\tilde{V}(i)$ - operator Focka
					\end{center}
					Schemat Algorytmu:
					\begin{enumerate}
						\item zakładamy $\Phi_{i}$ - każdy stan obsadzamy 2 elektronami
						\item Budujemy odpowiednio operatory kulombowski ($\tilde{J}_{q}(i)$) i wymiany($\tilde{K}_{q}(i)]$) :\\
						$\tilde{J}_{p}(i) \Psi_{q}(i)=[\int \Psi_{p}^{*}(j)\frac{e_{2}}{r_{ij}}\Psi_{p}(j)d\tau_{j}]\Psi_{q}(i)$
						
						$\tilde{K}_{p}(i) \Psi_{q}(i)=[\int \Psi_{p}^{*}(j)\frac{e_{2}}{r_{ij}}\Psi_{q}(j)d\tau_{j}]\Psi_{p}(i)$
						\item $F\Psi=\epsilon \Psi $ (duży epsilon)- otrzymujemy poziomy energetyczne 
						\item obliczamy energie całkowitą E
						\item $|E_{i}-E_{i-1}|<\Delta E$ jeśli tak to koniec ,a jeśli nie to  wróć do 2
					\end{enumerate}
					
					\subsection{rozwiązywanie równań HF przez rozwinięcie w bazie funkcji zlokalizowanych}
					
					KURWA ADAM ogarnij się
			\section{Wykład 9}
			\subsection{Wariacyjne Kwantowe Monte Carlo}
			Z metody wariacyjnej mamy funkcję falową wielocząstkową $\Psi\vec{\alpha}(\vec{r})$ gdzie $\vec{\alpha} = \{\alpha_1... ,\alpha_N\}$, $\vec{r} = \{r_1... ,r_N\}$. Energia całkowita: 
			$$<E> = \frac{1}{S} = \int\Psi\vec{\alpha}^*\vec{r}H\Psi\vec{\alpha}\vec{r}$$
			Kwantowe MC polega na minimalizacji <E>. Stosujemy algorytm metropolisa z użyciem bładzenia przypadkowego (random walker). Definiujemy energię lokalną:
			$$E_L(R) = \frac{H\Psi_T(R)}{\Psi_T(R)}$$
			$$\Psi_T^\alpha(r) \sim r$$
			$$<E> = \frac{\int dR \Psi^2_T(R)E_L(r)}{\int dR \Psi^2_T(R)}$$
			$$\Psi_T \in R$$
	        Konstruuje się algorytm typu metropolisa z rozkładu postaci:
	        $$\rho(R) = \frac{\Psi_T^2(R)}{\int dR'\Psi_r^2(R')}$$
	        \paragraph{Algorytm:}
	        \begin{itemize}
	        	\item ustawiamy N błądzących na losowych pozycjach: wybieramy błądzącego
	        	\item przesuwamy błądzącego na nową pozycję np. przesuwając pojedynczą cząstkę
	        	\item obliczamy prawdopodobieństwo $p=\frac{\Psi_T(R')}{\Psi_T(R)}$ gdzie R i R' to odpowiednio stara i nowa konfiguracja
	        	\item jeśli $p < 1$ to akceptujemy nową pozycje z prawdopodobieństwem, w przeciwnym wypadku akceptujemy gdy ilość kroków nie jest P większa od maksymalnego kroku
	        	\item wartość oczekiwana energii jest liczona jako średnia po próbach wygenerowanych w procedurze, np. dla oscylatora
	        \end{itemize}
	        \section{Wykład 10}
	        \subsection{Użycie AG dla optymalizacji struktur ciał stałych}
	        \paragraph{USPEX} przewidywanie struktur ciał stałych w oparciu o AG. Zagadnienie: znaleźć minimum dowolnej funkcji wielu parametrów. Metodą najmniejszego spadku nie trafimy do minimum - zagadnienie zwodniczości AG.
	        \subparagraph{Algorytm}
			\begin{itemize}
				\item inicjalizacja
				\item oceniamy przystosowanie każdego elementu populacji
				\item selekcja 
				\item operatory genetyczne 
				\item nowa populacja 
			\end{itemize} 			
			\subparagraph{CSP} - crystal structure prediction \\
			Mając skład stechiometryczny komórki elementarnej określić strukture najbardzeij stabilna energetycznie. Kombinatorycznie:
			$$C = \binom{V / \sigma^3}{V}\Pi_i \binom{N}{n_i}$$
			
\end{document} 